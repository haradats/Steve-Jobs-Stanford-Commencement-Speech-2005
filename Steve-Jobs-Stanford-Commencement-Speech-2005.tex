\documentclass[twocolumn]{jsarticle}

%\usepackage{doublespace}
%\usepackage[hypertext]{hyperref}
%\setlength{\parindent}{0pt} 
\setlength{\parskip}{1ex plus 0.5ex minus 0.2ex}

\usepackage{lmodern}
\usepackage[T1]{fontenc}
\usepackage{textcomp}

\begin{document}

\title{'You've got to find what you love,' Jobs says \\
ジョブズは言う,「愛するものを見つけるんだ」と}
\date{June 12, 2005\footnote{ジョブズがスタンフォード大学で講演を行った日.}
\footnote{本文書最終更新 \today}}
\author{Steve Jobs \\ 日本語訳: 原田季栄\footnote{NTTデータ先端技術}}

\maketitle



\section*{Just three stories}

I am honored to be with you today at your commencement from one of the finest universities in the world.
I never graduated from college.
Truth be told, this is the closest I've ever gotten to a college graduation.
Today I want to tell you three stories from my life.
That's it.
No big deal.
Just three stories.

\newpage

\section*{ただ三つの話}

世界でもっとも素晴らしい大学のひとつの卒業式に参列することを名誉に思います.
私は大学を卒業していません.
本当のことを言えば,これが私の人生で大学の卒業式にもっとも近づいた体験となります.今日,私は私の人生から3つの話をします.
それだけです.特にたいしたことはありません.
3つのお話だけです.

\newpage

\section*{The first story}

The first story is about connecting the dots.

I dropped out of Reed College after the first 6 months, but then stayed around as a drop-in for another 18 months or so before I really quit. So why did I drop out?

\vspace{\baselineskip}

It started before I was born. My biological mother was a young, unwed college graduate student, and she decided to put me up for adoption. She felt very strongly that I should be adopted by college graduates, so everything was all set for me to be adopted at birth by a lawyer and his wife. Except that when I popped out they decided at the last minute that they really wanted a girl. So my parents, who were on a waiting list, got a call in the middle of the night asking: ``We have an unexpected baby boy; do you want him?'' They said: ``Of course.''
My biological mother later found out that my mother had never graduated from college and that my father had never graduated from high school. She refused to sign the final adoption papers. She only relented a few months later when my parents promised that I would someday go to college.

\newpage

\section*{一つ目の話}

ひとつ目の話は,点をつなぐことについて.

私はリード大学を最初の6ヶ月でドロップアウトしました.
しかし,本当に辞めてしまうまでの18ヶ月間,部外者としてうろうろしていました.
では,どうして私はドロップアウトしたか(について説明します).

それは私が生まれる前からつながっています.私の生みの母は,若い未婚の大学卒業生でした.そして,彼女は私を養子に出すことを決意しました.彼女は,とても強く,私が大学を卒業した人に養子に出されるべきだと感じていました.それで,私が生まれたときには,弁護士とその妻の養子となるよう準備が整っていました.ただ,ひとつ.彼らは最後の最後に,女の子が欲しいと思ったことをのぞいては.それで,そのとき順番待ちのリストに登録されていた私の(現在の)両親は,夜中に電話をもらい,「予定外の男の子がいるけれど,希望しますか?」と聞かれました.彼らは,「もちろん」と答えました.
私の生母は,その後,私の母が大学を卒業しておらず,そして父が高校を卒業していないことを知りました.それで彼女は養子縁組に必要な書類,その最後の書類へのサインを拒否しました.数ヶ月後,私の両親が私を大学に行かせることを約束して,生母はやっと折れたのです.

\newpage

And 17 years later I did go to college. But I naively chose a college that was almost as expensive as Stanford, and all of my working-class parents' savings were being spent on my college tuition. After six months, I couldn't see the value in it. I had no idea what I wanted to do with my life and no idea how college was going to help me figure it out. And here I was spending all of the money my parents had saved their entire life. So I decided to drop out and trust that it would all work out OK. It was pretty scary at the time, but looking back it was one of the best decisions I ever made. The minute I dropped out I could stop taking the required classes that didn't interest me, and begin dropping in on the ones that looked interesting.

\vspace{\baselineskip}

It wasn't all romantic. I didn't have a dorm room, so I slept on the floor in friends' rooms, I returned Coke bottles for the 5¢ deposits to buy food with, and I would walk the 7 miles across town every Sunday night to get one good meal a week at the Hare Krishna temple. I loved it. And much of what I stumbled into by following my curiosity and intuition turned out to be priceless later on. Let me give you one example:

\newpage

そして17年後,私はちゃんと大学に行きました.しかし,私は世間知らずにもスタンフォード大学と同じくらい学費の高い大学を選びました(苦笑).そして,私の労働者階級の貯金が学費に費やされていきました.それから6ヶ月,私は大学に通うことの価値を見つけられませんでした.私は,人生でなにをしたいかわからず,大学がそれについてどう助けになるかわかりませんでした.そして,私は両親がその生涯に蓄えたすべての貯金を使い果たしていたのです.それで,私は大学をやめて,大丈夫なんとかなると自分に言い聞かせました.それは,そのときはとても恐ろしい体験でした,しかし,今振り返ると,私がこれまで人生で行ったきた判断の中でも最上に属するものでした.大学をやめた瞬間に,私は興味を感じない授業を受けなくてよくなりました.そして,興味を持った授業に顔を出すことを始めました.

ちっともロマンチックな状況ではありませんでした.私には寮の部屋がありませんでした.だから,友達の部屋の床に寝ていました.私は,コカコーラの空き瓶を5セントで引き取ってもらい,それを食費に充てていました.毎週,日曜日の夜には7マイル\footnote{約11キロ}歩き,ハレ・クリシュナ寺に行き,おいしい食事を食べていました.それは格別でした.そして,後になって,私が好奇心と直感に引かれ出くわしたものの多くが,かけがえのない価値を持つものでした.その例について話します.

\newpage

Reed College at that time offered perhaps the best calligraphy instruction in the country. Throughout the campus every poster, every label on every drawer, was beautifully hand calligraphed. Because I had dropped out and didn't have to take the normal classes, I decided to take a calligraphy class to learn how to do this. I learned about serif and sans serif typefaces, about varying the amount of space between different letter combinations, about what makes great typography great. It was beautiful, historical, artistically subtle in a way that science can't capture, and I found it fascinating.

\vspace{\baselineskip}

None of this had even a hope of any practical application in my life. But 10 years later, when we were designing the first Macintosh computer, it all came back to me. And we designed it all into the Mac. It was the first computer with beautiful typography. If I had never dropped in on that single course in college, the Mac would have never had multiple typefaces or proportionally spaced fonts. And since Windows just copied the Mac, it's likely that no personal computer would have them. If I had never dropped out, I would have never dropped in on this calligraphy class, and personal computers might not have the wonderful typography that they do. Of course it was impossible to connect the dots looking forward when I was in college. But it was very, very clear looking backward 10 years later.

\newpage

リード大学はそのころ,おそらくアメリカでもっとも優れたカリグラフィーの指導を行っていました.学内では,あらゆるポスター,引き出しに貼られた見出しなど,美しく手書きされていました.私は,落ちこぼれていて,通常の授業を受ける必要がなかったので,そうした美しい手書きのやり方を覚えるためにカリグラフィーの授業を受けることにしました.私は,serifとsans serifの書体について,文字の組み合わせに応じて,文字の間の空白を調整することについて,また何が美しい書体を作るのかについて学びました.それは,美しく,歴史の流れを汲んでおり,科学の到達できない領域で美的に洗練されていました.そして,私は強く惹かれました.

それらのどれについても,私のそれからの人生において現実的な応用ができるとは思えませんでした.しかし,それから十年が経過して,私たちがマッキントッシュを設計していたとき,それらすべてが私のところに返ってきました.
私たちはそれらをそっくりマックに組み込みました.マックは美しい書体に対応した最初のコンピュータでした.
もし,私が落ちこぼれていなければ,マッキントッシュは,複数の書体や文字間のスペースを調整する機能を持たなかったでしょう.ウィンドウズは,マックをパクったので,パーソナルコンピュータはそれらを持たなかったことでしょう.もし,私が落ちこぼれていなければ,私はカリグラフィーのクラスに顔を出すことはなく,パーソナルコンピュータは,現在のような素晴らしい書体を持たなかったことでしょう.
もちろん,私が大学にいたころに,点を未来につなげることはできませんでした.しかし,10年経過してから振り返ったとき,点は非常に明確につながっていたのです.


\newpage

Again, you can't connect the dots looking forward; you can only connect them looking backward. So you have to trust that the dots will somehow connect in your future. You have to trust in something — your gut, destiny, life, karma, whatever. This approach has never let me down, and it has made all the difference in my life.

\newpage

もう一度言います.点を未来につなげることはできません.過去を振り返ることによってのみ点をつなげることができるのです.だから,あなた方は点が未来にどうにかしてつながるということを信じる必要があります.何かに信を置かなければなりません.勇気,運命,人生,カルマ,その他なんであったとしても.この方法をとることによって,私は一度も落ち込むことがありませんでした.そして,人生をまったく変えたのです.

\newpage

\section*{The second story}

My second story is about love and loss.

I was lucky — I found what I loved to do early in life. Woz\footnote{Steve Wozniak} and I started Apple in my parents' garage when I was 20. We worked hard, and in 10 years Apple had grown from just the two of us in a garage into a \$2 billion company with over 4,000 employees. We had just released our finest creation — the Macintosh — a year earlier, and I had just turned 30. And then I got fired. How can you get fired from a company you started? Well, as Apple grew we hired someone who I thought was very talented to run the company with me, and for the first year or so things went well. But then our visions of the future began to diverge and eventually we had a falling out. When we did, our Board of Directors sided with him. So at 30 I was out. And very publicly out. What had been the focus of my entire adult life was gone, and it was devastating.

\vspace{\baselineskip}
\vspace{\baselineskip}

I really didn't know what to do for a few months. I felt that I had let the previous generation of entrepreneurs down — that I had dropped the baton as it was being passed to me. I met with David Packard and Bob Noyce\footnote{Robert Norton Noyce} and tried to apologize for screwing up so badly. I was a very public failure, and I even thought about running away from the valley. But something slowly began to dawn on me — I still loved what I did. The turn of events at Apple had not changed that one bit. I had been rejected, but I was still in love. And so I decided to start over.

\newpage

\section*{二つ目の話}

二つ目の話は,愛と喪失についてです.


私は幸運でした.人生の早い段階で,熱中できることを見つけました.ウォズと私は,私が20歳のときに,私の家の庭でアップルを始めました.私たちは一生懸命がんばって,10年後アップルは庭の二人の会社から,従業員4000人,年商20億ドルの会社にまで成長しました.創業10周年を迎える一年前,私たちは自分たちが創ることができた最上のもの,マッキントッシュを世に送り出しました.そして,私は30歳になり,会社をクビになりました.いったい,どうしたら自分が始めた会社でクビにされるのでしょう?(苦笑)アップルの経営の規模が大きくなったので,私は一緒に会社を経営する才能があると思われる人を雇いました.そして,それは最初の年とちょっとはうまくいきました.しかし,未来のビジョンについて,意見を異にするようになり,最後は決裂しました.そのとき,ボードメンバーやディレクター達は彼のほうを支持しました.それで,30歳のとき,私は会社を出されました.それも非常に派手で目立つ形で.私の成人してからの人生全体の焦点が失われ,それは破壊的でした.

私は数ヶ月の間,本当にどうして良いかわかりませんでした.私は,私の前のアントレプラナー達をがっかりさせた,私に託されたバトンを落としてしまった,と思いました.私は,デイビッド・パッカードとボブ・ノイスに会って,こんなにもひどく台無しにしたことを謝ろうとしました.私の失敗は広く知られており,私はシリコンバレーから逃げ出すことすら考えました.しかし,何かがゆっくりと私に広がってきました.私は,まだ私がやっていたことに愛を感じていました.アップルで起こった出来事は,ただの1ビットもそれを変えていませんでした.私は追い出されました,しかし,私の愛は続いていました.そして,私はまた始めようと決意しました.

\newpage

I didn't see it then, but it turned out that getting fired from Apple was the best thing that could have ever happened to me. The heaviness of being successful was replaced by the lightness of being a beginner again, less sure about everything. It freed me to enter one of the most creative periods of my life.

During the next five years, I started a company named NeXT, another company named Pixar, and fell in love with an amazing woman who would become my wife. Pixar went on to create the world's first computer animated feature film, \textit{Toy Story}, and is now the most successful animation studio in the world. In a remarkable turn of events, Apple bought NeXT, I returned to Apple, and the technology we developed at NeXT is at the heart of Apple's current renaissance. And Laurene and I have a wonderful family together.

\newpage

そのときはわかりませんでした.しかし,アップルをクビにされたことは,私について起こりえた最上の出来事でした.成功したことに伴う重圧は,再び何もわからない初心者の気軽さにとって変わりました.そのことは,私を人生で再びもっとも創造的な時期に入るために解放してくれました.

\vspace{\baselineskip}

それからの5年間,私はNeXT,そしてPixarと呼ばれる会社を立ち上げました.そして,現在私の妻である驚嘆すべき女性と出会い,恋に落ちました.ピクサーは,世界で最初のコンピュータによるアニメーション,「トイ・ストーリー」を創ろうとしており,今や世界でもっとも成功したアニメーションスタジオです.驚くべき巡り合わせの結果,アップルはNeXTを買収し,私はアップルに戻りました.そして,私たちがNeXT社で開発した技術は,アップルの現在にいたる復興の原動力になっています.そして,ローリーンと私は,素晴らしい家庭を持つことができました.

\newpage

I'm pretty sure none of this would have happened if I hadn't been fired from Apple. It was awful tasting medicine, but I guess the patient needed it. Sometimes life hits you in the head with a brick. Don't lose faith. I'm convinced that the only thing that kept me going was that I loved what I did. You've got to find what you love. And that is as true for your work as it is for your lovers. Your work is going to fill a large part of your life, and the only way to be truly satisfied is to do what you believe is great work. And the only way to do great work is to love what you do. If you haven't found it yet, keep looking. Don't settle. As with all matters of the heart, you'll know when you find it. And, like any great relationship, it just gets better and better as the years roll on. So keep looking until you find it. Don't settle.

\newpage

これらのことすべてについて,私がアップルをクビにならなければ起こらなかったのは明白です.疑う余地はありません.それはひどい味の薬でした,しかし,思うに患者(である自分)はそれを必要としていたのです.人生には,ときどき,煉瓦で頭を殴られるような出来事が起こります.(そのとき)信じることをやめてはいけません.
私は,自分がこれまで続けることを可能にした唯一の力は,私が自分のしていたことを愛していたからだと納得しています.
あなた方は愛せるものを見つけなければいけません.
そして,そのことはあなたたちの恋人に対してそうであるように,仕事に対しても成り立ちます.仕事はあなた方の人生において大きな位置を占めるものであり,真に満足するための唯一の方法は,自分で偉大だと思える仕事をなすことです.
そして,偉大な仕事を成し遂げる唯一の方法は,自分がやっていることを愛することなのです.
もし,あなたがまだそれを見つけていないなら,探し続けることです.ぼけっとしていてはいけません.あなたの心で起こるすべてと同様に,あなたはそれを発見すると自分で気がつきます.そして,あらゆる最上の関係がそうであるように,年月を重ねることにより深まっていきます.だから,探し続けてください.ぼんやりしないで.

\newpage

\section*{The third story}

My third story is about death.

When I was 17, I read a quote that went something like: ``If you live each day as if it was your last, someday you'll most certainly be right.'' It made an impression on me, and since then, for the past 33 years, I have looked in the mirror every morning and asked myself: ``If today were the last day of my life, would I want to do what I am about to do today?'' And whenever the answer has been ``No'' for too many days in a row, I know I need to change something.

Remembering that I'll be dead soon is the most important tool I've ever encountered to help me make the big choices in life. Because almost everything — all external expectations, all pride, all fear of embarrassment or failure — these things just fall away in the face of death, leaving only what is truly important. Remembering that you are going to die is the best way I know to avoid the trap of thinking you have something to lose. You are already naked. There is no reason not to follow your heart.



\newpage

\section*{三つ目の話}

私の三番目の話は,死についてです.

私が17歳のとき,「もし毎日を「今日が人生最後の日だ」という気持ちで生きていれば,いつかそれが本当になる日がくる」,そのような引用を読んだことがあります.それは私の印象に残り,それからの33年間,私は毎朝鏡を見て自問してきました,「もし今日が人生最後の一日だとしたら,自分は今日予定していることをしたいだろうか?」,と.そして,その答えが「いや,違う」が何日も続くとき,私はいつも何かを変えなければいけないのだと気づくのです.



自分がじきにこの世を去ることを思い出すことは,人生において大きな変更を行う際の,私が知る限りもっとも重要なツールです.なぜなら,あらゆる周囲の期待,あらゆる誇り,あらゆる困惑や失敗へのおそれ,そうしたものは,ほとんどすべてが死を前にして崩れ去り,真に大切なもののみが残るからです.自分たちがやがて死ぬということを思い出すことは,私が知る限り「何か失うものがあるかもしれないと思いこむ」罠からあなた方を救う最上の方法です.あなた方は,すでに丸裸です.心の思うままに従わない理由など,どこにもありません.



\newpage

About a year ago I was diagnosed with cancer. I had a scan at 7:30 in the morning, and it clearly showed a tumor on my pancreas. I didn't even know what a pancreas was. The doctors told me this was almost certainly a type of cancer that is incurable, and that I should expect to live no longer than three to six months. My doctor advised me to go home and get my affairs in order, which is doctor's code for prepare to die. It means to try to tell your kids everything you thought you'd have the next 10 years to tell them in just a few months. It means to make sure everything is buttoned up so that it will be as easy as possible for your family. It means to say your goodbyes.

I lived with that diagnosis all day. Later that evening I had a biopsy, where they stuck an endoscope down my throat, through my stomach and into my intestines, put a needle into my pancreas and got a few cells from the tumor. I was sedated, but my wife, who was there, told me that when they viewed the cells under a microscope the doctors started crying because it turned out to be a very rare form of pancreatic cancer that is curable with surgery. I had the surgery and I'm fine now.

This was the closest I've been to facing death, and I hope it's the closest I get for a few more decades. Having lived through it, I can now say this to you with a bit more certainty than when death was a useful but purely intellectual concept:

\newpage

一年くらい前,私はガンだと告げられました.朝,7時30分に撮影を行ったところ,膵臓にはっきり腫瘍が写っていました.私は膵臓が何かすら知りませんでした.主治医たちは,私にこれはほぼ治療できないガンと思われ,余命は3ヶ月あるいは6ヶ月に及ばないと言いました.
主治医は,家に行って,用事を片づけておくようにと言いました.それは,「死に備えなさい」というお医者さんの暗号です.
それは,もしこれから10年生きていたら子供たちに伝えるであろうすべてのことを,たった数ヶ月の間に伝えなさいということを意味します.
残される家族が楽になるように,自分ができるあらゆることをしておけということです.
別れを告げなさい,そういうことです.


\vspace{\baselineskip}

私は終日その診断のことを考えていました.その夜,遅く,喉から胃,腸までへと内視鏡を通し,針のようなものでガンの細胞を採取しました.
私は麻酔でもうろうとして記憶がありません.しかし,つきそってくれていた妻があとで,「お医者さんが顕微鏡で採取した細胞を調べたら,それがきわめてまれな外科手術で治療できるものだとわかって泣きだしたのよ」と教えてくれました.
私は手術を受け,そして今大丈夫です.

\vspace{\baselineskip}
\vspace{\baselineskip}

これは私にとってもっとも死に近い経験でした.そして,私はあと数十年,それ以上のことがないことを望んでいます.
この経験があるので,私は今,あなたがたに,死が純粋に知的な概念であった頃より少しは明確に,このことをあなた方に言えるのです.

\newpage

No one wants to die. Even people who want to go to heaven don't want to die to get there. And yet death is the destination we all share. No one has ever escaped it. And that is as it should be, because Death is very likely the single best invention of Life. It is Life's change agent. It clears out the old to make way for the new. Right now the new is you, but someday not too long from now, you will gradually become the old and be cleared away. Sorry to be so dramatic, but it is quite true.

\vspace{\baselineskip}


Your time is limited, so don't waste it living someone else's life. Don't be trapped by dogma — which is living with the results of other people's thinking. Don't let the noise of others' opinions drown out your own inner voice. And most important, have the courage to follow your heart and intuition. They somehow already know what you truly want to become. Everything else is secondary.

\newpage

誰も死にたくありません.天国にいきたいと願う人ですら,死にたくないでしょう.でも,死は我々すべてに共通する到着点です.誰もそこから逃げられません.そして,またそうであるべきなのです.なぜなら,死こそが人生における唯一最上の発明と思われるからです.死は,人生を変える手段です.
それは,新しい人々を受け入れるために老人を整理します.
今この瞬間は,あなた方がその若者です.
しかし,それほど先でない近い将来,あなた方は徐々に年をとり,そしてこの世を去ります.
芝居じみてすみません,しかしこれはまったく本当のことなのです.

あなたの時間には限りがあります.だから,それを他の誰かの人生のために無駄にしてはいけません.ドグマ,他の人考えた結果,にとらわれてはいけません.他の人の考えの雑音に,あなたの内なる声をおぼれさせてはいけません.そして,もっとも重要なこと,それはあなたの心と直感にしたがう勇気を持つことです.あなたの心と直感,それらはどうかして,あなたが真になりたいことを知っています.それら以外は重要ではありません.

\newpage


\section*{Stay hungry. Stay foolish.}

When I was young, there was an amazing publication called \textit{The Whole Earth Catalog}, which was one of the bibles of my generation. It was created by a fellow named Stewart Brand not far from here in Menlo Park, and he brought it to life with his poetic touch. This was in the late 1960s, before personal computers and desktop publishing, so it was all made with typewriters, scissors and Polaroid cameras. It was sort of like Google in paperback form, 35 years before Google came along: It was idealistic, and overflowing with neat tools and great notions.

\vspace{\baselineskip}
\vspace{\baselineskip}


Stewart and his team put out several issues of \textit{The Whole Earth Catalog}, and then when it had run its course, they put out a final issue. It was the mid-1970s, and I was your age. On the back cover of their final issue was a photograph of an early morning country road, the kind you might find yourself hitchhiking on if you were so adventurous. Beneath it were the words: ``Stay Hungry. Stay Foolish.'' It was their farewell message as they signed off. Stay Hungry. Stay Foolish. And I have always wished that for myself. And now, as you graduate to begin anew, I wish that for you.

Stay Hungry. Stay Foolish.

Thank you all very much.

\newpage

\section*{空腹であれ,愚か者であれ}

私が若かったころ,「ホールアースカタログ」と呼ばれる驚嘆すべき雑誌があり,それは私たちの世代の聖典のひとつでした.それは,ここからあまり離れていないメンロ・パークに住んでいた,スチュワート・ブランドと呼ばれる若者により創られたもので,彼の詩的なセンスを吹き込まれていました.まだパーソナルコンピュータやデスクトップパブリッシングのなかった1960年代後半のことなので,すべてがタイプライター,はさみ,そしてポラロイドカメラで造られていました.言ってみれば,Googleをペーパーバックにしたようなもので,Googleが登場する35年前でした.観念主義的であり,気の利いたツールと素晴らしい言い回しが満ち溢れていました.

スチュワートと彼のチームは何冊かのホールアースカタログを発刊し,そしてその旅を終えたとき,最終号を発刊しました.1970年代の中頃で,私があなた方くらいの年代のころです.最終号の裏表紙には,朝のカントリーロードの写真があり,冒険心を持つ読者であれば,自分がヒッチハイクしているようなつもりにさせるものでした.下の方に言葉が書かれていました.「空腹であれ,愚か者であれ」.それは,雑誌を終了する彼らからのお別れのメッセージでした.空腹であれ,愚か者であれ,と.私はいつも自分がそうありたいと願いながら生きてきました.そして今,あなた方が巣立つときなので,私はあなた方にそれを願います.

空腹であってください,愚か者であってください.

どうもありがとうございました.

\newpage

\section*{おわりに}


本資料で紹介したスタンフォード大学での講演および,
ジョブズが行ったキーノートスピーチの動画について,
AppleのiTunes Music Storeで配信されています.
新しいタイトルが追加されることはありませんが,残された
映像の価値はあせることなく,語り継がれることでしょう.
私がTOMOYO Linux\footnote{http://tomoyo.sourceforge.jp/}の仕事で海外で発表を
行うことになったとき,
繰り返しジョブズの動画を観て,スピーチを聞いたことを思い出します.
是非,原語で「プレゼンテーションの名手」,ジョブズに
接し,そして学んでみて欲しいと思います.

\appendix


\section{本文書について}

2015年10月23日,都内のある大学の学生さん達に「人生をより良く生きるための
プレゼンテーション入門」\footnote{http://haradats.blogspot.jp/p/index.html}と題して講義を行いました.講義では,ジョブズが
2005年に行った伝説の講演を素材のひとつとして取り上げ,4bitプロセッサ,
i4004発表に始まるパーソナルコンピュータ黎明期の出来事を振り返りながら,
私なりの解釈について説明を行いました.本文書は,その講義の配布用資料として作成したものです.

講演は,
すでに世界中の言語に翻訳され公開されており,日本語のものもありますが,
ジョブズへの感謝と敬意を込めて,今回新たに翻訳を行い\LaTeX を用いて文書を作成しました.
この文書がまたdotとなり,将来につながることを信じて.

\section{講演テキストおよび本文書の構成について}

本文書の内容は,スティーブ・ジョブズが2005年6月12日にスタンフォード大学の
卒業式で行った講演の内容に基づいています.
知る限りジョブズ自身としては,講演の原稿を公開していませんが,
スタンフォード大学では,ジョブズの講演について,
Stanford Report (June 14, 2005)\footnote{http://news.stanford.edu/news/2005/june15/jobs-061505.html}として,文字興しした内容と動画を公開しています.
講演内容について,英語の部分はそのWebページのものを
無修正で用いています.

本文書のタイトルは,オリジナルであるWebページのタイトルを使わせていただきました.
約15分の講演の内容を読みやすくする意図のもと,
全体を5つの節に分割しています.節ごとにタイトルが必要となりましたが,
講演は,「3つの話(just three stories)」を軸としているため,
3つの話については,それぞれ一つ目,二つ目,三つ目の話とし,
冒頭は「3つの話」,最後はジョブズが学生に送ったメッセージである
「Stay Hungry, Stay Foolish」としました.

ジョブズのスピーチは,比較的平易な単語で書かれています.
書いたその人自身のようにシンプルでストレートでありながら,
不思議な魅力と余韻があります.原文を味わって欲しいとの思いから,
本文は2カラム構成とし,
原文と日本語訳を並べるレイアウトとしました.\footnote{岡村晴彦先生の
「\LaTeX2e 美文書作成入門」を参考とさせていただきました.}



%\section*{文書公開url}
%\verb|http://lab.iisec.ac.jp/~tanaka_lab/images/stanford.pdf|

%\section*{更新日}
%\today

\end{document}